\documentclass[a4paper,14pt]{extarticle}
\usepackage[utf8]{inputenc}

\usepackage[T2A]{fontenc}
\usepackage[utf8]{inputenc}
\usepackage[russian]{babel}

\usepackage{multienum}
\usepackage{geometry}
\usepackage{hyperref}
\usepackage{mathptmx}

\usepackage{cyrtimes}

% настройка ссылок и метаданных документа
\hypersetup{unicode=true,colorlinks=true,linkcolor=black,citecolor=green,filecolor=magenta,urlcolor=cyan,
    pdftitle={Конфликты в обществе},
    pdfauthor={Сидорук Данил Вадимович},
    pdfsubject={Этика делового общения},
    pdfcreator={Сидорук Данил Вадимович},
    pdfproducer={Overleaf},
    pdfkeywords={Конфликты в обществе, этика делового общения}
}

\geometry{top=2cm}
\geometry{bottom=2cm}
\geometry{left=3cm}
\geometry{right=1cm}
\geometry{bindingoffset=0cm}

\makeatletter
\renewenvironment{titlepage} {
    \thispagestyle{empty}
}
\makeatother
\usepackage{indentfirst}

\title{Конфликты в коллективе}
\author{Сидорук Данил Вадимович}
\date{6 апреля 2023 г.}

\newtheorem{definition}{Определение}
\usepackage{setspace}

\begin{document}
\raggedright
\setlength{\parindent}{1.25cm} % для шрифта 14pt

\begin{titlepage}   % начало титульной страницы

\begin{center}
\textsc{МИНИСТЕРСТВО ЦИФРОВОГО РАЗВИТИЯ СВЯЗИ И МАССОВЫХ КОММУНИКАЦИЙ\\[2mm]
Ордена Трудового Красного Знамени\\[2mm]
Федеральное государственное бюджетное учреждение высшего образования\\[2mm]
«Московский технический университет связи и информатики»\\[5mm]
Кафедра «Философия, история и межкультурные коммуникации»}\\[40mm]

\textbf{Реферат}\\[2mm]
по дисциплине «Этика делового общения» по теме:\\[2mm]
Конфликты в коллективе
\\[20mm]
\end{center}

\vfill

\hfill
\begin{minipage}{.5\textwidth}
Выполнил: студент группы БФИ2202 \ Сидорук Д. В.\\[2mm]
Проверила: доц., канд. ист. наук Скляр Л. Н.
\end{minipage}%
\vfill
\begin{center}
 Москва, 2023 г.
\end{center}

\end{titlepage}

\pagebreak
\tableofcontents
\pagebreak

\onehalfspacing
\section{Введение}

Проблема исследования конфликтов является междисциплинарной и весьма актуальной в настоящее время, как для ученых, так и для практических психологов, социологов, руководителей любого ранга. Причина заключается в том, что конфликты возникают постоянно практически во всех сферах человеческой жизни. Несмотря на многочисленные исследования по вопросам возникновения, функционирования, разрешения конфликтов, попытки классификации конфликтов по различным основаниям, общепринятой теории конфликтов, объясняющей их природу, влияние на развитие личности, коллектива, общества, нет.\par

\hfill

Учение о конфликтах берет свое начало в трудах древнегреческих ученых — основоположников этики — Сократа, Платона, Гераклита, находит развитие в учении о противоречиях и борьбе противоположностей Г. Гегеля. Впервые попытка рассмотреть конфликт как системное и необходимое явление культуры в целом предпринята Г. Зиммелем, который писал о непрерывной изменчивости культурных явлений и стилей, о противоречии, результатом которого является зримое вытеснение старых форм новыми. Дальнейшие подходы рассматривали понятие социального конфликта как борьбы за ценности и притязания на статус, власть и ресурсы, в ходе которой оппоненты нейтрализуют,
наносят ущерб или устраняют своих соперников.

\pagebreak
\section{Понятие конфликта, его последствия, стадии, типы и структура}

\begin{definition}
\textbf{Конфликт} — это открытое противостояние, являющееся следствием взаимоисключающих интересов и позиций
\end{definition}

Основу конфликтных ситуаций в группе между отдельными людьми составляет столкновение между противоположными интересами, мнениями, целями, различными представлениями о способе их достижения.

В социальной психологии существует многовариантная типология конфликта в зависимости от тех критериев, которые берутся за основу.

\hfill

Так, например, конфликт может быть внутриличностным (между родственными симпатиями и чувством служебного долга руководителя); межличностным (между руководителем и его заместителем по поводу должности, премии между сотрудниками); между личностью и организацией, в которую она входит; между организациями или группами одного или различного статуса.

\hfill

Возможны также классификации конфликтов по горизонтали (между рядовыми сотрудниками, не находящимися в подчинении друг к другу), по вертикали (между людьми, находящимися в подчинении друг к другу) и смешанные, в которых представлены и те, и другие. Наиболее распространены конфликты вертикальные и смешанные. Они в среднем составляют 70-80\% от всех конфликтов, являются нежелательными для руководителя, так как в них он как бы "связан по рукам и ногам". Дело в том, что в этом случае каждое действие руководителя рассматривается всеми сотрудниками через призму этого конфликта.

\hfill

Конфликты могут явиться результатом недостаточного общения и понимания, неверных предположений в отношении чьих-либо действий, различий в планах, интересах и оценках.

Допустима также классификация по характеру причин, вызвавших конфликт. Перечислить все причины возникновения конфликта не представляется возможным. Но в целом он вызывается, как указывает Р.Л. Кричевский в книге «Если вы — руководитель ...» тремя группами причин, обусловленными:

\begin{enumerate}
    \item трудовым процессом;
    \item психологическими особенностями человеческих взаимоотношений, то есть их симпатиями и антипатиями, культурными, этническими различиями людей, действиями руководителя, плохой психологической коммуникацией и т.д.;
    \item личностным своеобразием членов группы, например, неумением контролировать свое эмоциональное состояние, агрессивностью,- некоммуникабельностью, бестактностью и тд.
\end{enumerate}

Конфликты различают и по их значению для организации, а также по способу их разрешения. Различают конструктивные и деструктивные конфликты.

Для конструктивных конфликтов характерны разногласия, которые затрагивают принципиальные стороны, проблемы жизнедеятельности организации и ее членов и разрешение которых выводит организацию на новый более высокий и эффективный уровень развития.

Деструктивные конфликты приводят к негативным, часто разрушительным действиям, которые иногда перерастают в склоку и другие негативные явления, что резко снижает эффективность работы группы или организации.

\subsection{Стадии конфликта}

Несмотря на свою специфику и многообразие, конфликты имеют в целом общие стадии протекания:

\begin{enumerate}
    \item потенциальное формирование противоречивых интересов, ценностей, норм
    \item переход потенциального конфликта в реальный или стадию осознания участниками конфликта своих верно или ложно понятых интересов
    \item конфликтные действия
    \item снятие или разрешение конфликта
\end{enumerate}

\subsection{Структура конфликта}

Кроме того, каждый конфликт имеет также более или менее четко выраженную структуру. В любом конфликте присутствует объект конфликтной ситуации, связанный либо с организационными и технологическими трудностями, особенностями оплаты труда, либо со спецификой деловых и личных отношений конфликтующих сторон.

Следующий элемент конфликта — цели, субъективные мотивы его участников, обусловленные их взглядами и убеждениями, материальными и духовными интересами.

Далее, конфликт предполагает наличие оппонентов, конкретных лиц, являющихся его участниками.

И, наконец, в любом конфликте важно отличить непосредственный повод столкновения от подлинных его причин, зачастую скрываемых.

\hfill

Руководителю-практику важно помнить, что пока существуют все перечисленные элементы структуры конфликта (кроме повода), он неустраним. Попытка прекратить конфликтную ситуацию силовым давлением либо уговорами приводит к нарастанию, расширению его за счет привлечения новых лиц, групп или организаций. Следовательно, необходимо устранить хотя бы один из существующих элементов структуры конфликта.

Специалистами разработано немало рекомендаций, касающихся различных аспектов поведения людей в конфликтных ситуациях, выбора соответствующих стратегий поведения и средств разрешения конфликта, а также управления им.

\subsection{Последствия конфликтов}

Негативные последствия конфликтов:

\begin{multienumerate}
    \mitemxxx{Физическая и духовная усталость}{Ухудшение взаимоотношений}{Снижение функциональности}
\end{multienumerate}

Позитивные последствия конфликтов:

\begin{enumerate}
    \item Конфликт является источником развития
    \item Дает возможность сформулировать и осознать проблему, выявить трудности
    \item Конфликт помогает найти новые пути решения
    \item Учит формам общения, умению налаживать отношения
    \item Оказывает содействие личностному, профессиональному росту и развивает волевую саморегуляцию
    \item Конфликт — это возможность разрядки напряжения, «оздоровления» отношений
\end{enumerate}

\pagebreak
\section{Предпосылки возникновения конфликта в процессе общения}

Рассмотрим особенности поведения человека в конфликтной ситуации прежде всего в процессе общения. В процессе человеческих взаимоотношений процесс общения предполагает наличие следующих трех факторов: восприятия, эмоций и обмена информацией. В конфликтных ситуациях легко забыть об этом. Поэтому кратко рассмотрим, что же может создавать почву для их возникновения.

\textbf{Социально-психологические предпосылки} Первая трудность — это разногласия из-за несовпадения ваших рассуждении с рассуждениями другой стороны. Ведь то, какой вы видите проблему, зависит от того, с какой колокольни, образно говоря, смотрите на нее. Люди склонны видеть то, что хотят видеть. Из массы фактов мы изымаем те, которые подтверждают наши взгляды, представления и убеждения, и не обращаем внимания или ошибочно интерпретируем те из них, которые ставят под вопрос наши представления. Однако следует иметь в виду, что понять точку зрения другого — это еще не значит согласиться с ней. Это может помочь лишь сузить область конфликта.

Также не следует интерпретировать высказывания или действия другой стороны в негативном плане, так как это вызывает отрицательные эмоции. Но на отрицательные эмоции в свой адрес мы испытываем раздражение и у нас возникает желание компенсировать свой психологический проигрыш, ответив обидой на обиду. При этом ответ должен быть не слабее, и для уверенности он делается с "запасом". Снисходительное отношение, категоричность, подшучивание, напоминание о какой-то проигранной ситуации и т.п. — все это вызывает отрицательную реакцию у окружающих и служит питательной средой для возникновения конфликтной ситуации.

\textbf{Непонимание и нежелание слушать} Следующая трудность, которая возникает в процессе общения и может оказывать влияние на возникновение конфликта, это то, что люди, очень часто разговаривая, не понимают друг друга. Даже если вы говорите ясно и прямо, вас могут не услышать. Как часто вам кажется, что люди не обращают внимание на ваши слова. Столь же часто и вы не в состоянии повторить то, что они сказали, так как в этот момент можете быть заняты обдумыванием контраргумента и т.п. Кроме того то, что говорит один, другой может не так понять. Все это вместе взятое и создает предпосылки к конфликту и трудности в управлении им.

Учитывая трудности в процессе общения, Е.  Мелибрудой, В. Зигертом и Л. Ланге, была разработана модель поведения человека в конфликтной ситуации с точки зрения ее соответствия психологическим стандартам. Считается, что конструктивное разрешение конфликта зависит от следующих факторов:

\begin{enumerate}
    \item адекватности восприятия конфликта, то есть достаточно точной, не искаженной личными пристрастиями оценки поступков, намерений как противника, так и своих собственных;
    \item открытости и эффективности общения, готовности к всестороннему обсуждению проблем, когда участники честно высказывают свое понимание происходящего и пути выхода из конфликтной ситуации;
    \item создания атмосферы взаимного доверия и сотрудничества.
\end{enumerate}

Для руководителя также полезно знать, какие индивидуальные особенности личности (черты характера) создают у человека склонность или предрасположенность к конфликтным отношениям с другими людьми. Обобщая исследования психологов, можно сказать, что к таким качествам относятся:

\begin{enumerate}
    \item неадекватная самооценка своих возможностей и способностей, которая может быть как завышенной, так и заниженной. И в том, и в другом случае она может противоречить адекватной оценке окружающих — и почва для конфликта готова;
    \item стремление доминировать во что бы то ни стало там, где это возможно и невозможно; сказать свое последнее слово;
    \item консерватизм мышления, взглядов, убеждений, нежелание преодолеть устаревшие традиции;
    \item излишняя принципиальность и прямолинейность в высказываниях и суждениях, стремление во что бы то ни стало сказать правду в глаза;
    \item критический настрой, особенно необоснованный и не аргументированный;
    \item определенный набор эмоциональных качеств личности — тревожность, агрессивность, упрямство, раздражительность.
\end{enumerate}

Но конфликт возникает, если личностные особенности человека или группы приходят в столкновение с вышеназванными особенностями человека, предрасположенного к конфликтам, т.е. при наличии межличностной или социально-психологической несовместимости.

В качестве примера рассмотрим несовместимые типы темперамента при определенных условиях. В нормальной спокойной    обстановке холерик и флегматик успешно справляются с порученной им работой. В аварийной ситуации медлительность флегматика, желание обдумать ход деятельности и вспыльчивость, неуравновешенность и суетливость холерика могут стать причиной конфликтных отношений между ними.

Еще более часто основой для межличностной несовместимости становятся различия в потребностях, интересах, целях  различных людей, вступающих во взаимодействие. Основной интерес, например у руководителя образованной фирмы или предприятия — расширить дело, а у сотрудников  — чтобы как можно больше средств было выделено на зарплату. Это создает трения между ними, которые могут привести к конфликту даже близких людей.

Социально-психологическая несовместимость может также возникнуть из-за того, что группа, окружение предъявляет личности требования, которые расходятся с теми, на которые ориентирован этот человек.

\pagebreak
\section{Способы разрешения конфликтов}

Разрешение конфликта связано с изменением конфликтной ситуации, а способы разрешения — со способами изменения конфликтной ситуации.

Важно помнить, что:

\begin{enumerate}
    \item Неразрешенный конфликт порождает новые конфликтные ситуации с новыми оппонентами
    \item Между оппонентами возникает и укрепляется чувство неприязни, они превращаются в противников
\end{enumerate}

Ученые выделяют два основных типа методов решения конфликтов: прямые (открытые) и косвенные (скрытые).

\subsection{Прямые, или открытые, пути разрешения конфликтов}

\begin{enumerate}
    \item \textbf{Директивный} — настойчивое утверждение своей точки зрения. Этот метод не самый эффективный, но зато именно к нему и прибегает большинство людей.

    Он используется лишь в следующих случаях:
    \begin{enumerate}
        \item Вы обладаете несомненной властью и авторитетом, а предлагаемое решение — наилучшее
        \item Вы чувствуете, что нет выбора и вам нечего терять
        \begin{enumerate}
            \item У вас достаточно полномочий, чтобы принять необходимое, но непопулярное решение
            \item Если взаимодействуете с подчиненными, предпочитающими авторитарный стиль
        \end{enumerate}
    \end{enumerate}
    \item \textbf{Сотрудничество} - находится наиболее приемлемое для конфликтующих сторон решение, а оппоненты превращаются в партнеров. Это один из самых сложных путей, но самых эффективных.

    Он используется в следующих случаях:
    \begin{enumerate}
        \item Существуют длительные и прочные отношения между сторонами конфликта
        \item Имеется общая точка зрения на конфликт
    \end{enumerate}
    \item \textbf{Игнорирование конфликта}. Такой путь вовсе не означает вашего поражения. Вы просто сглаживаете атмосферу, не пытаясь отстаивать свои интересы, выжидаете более благоприятного момента, чтобы решить свою проблему

    Он используется в следующих случаях:
    \begin{enumerate}
        \item Источник разногласий тривиален и несущественен по сравнению с более важными задачами
        \item Слишком велика напряженность, и необходимо время, чтобы восстановить спокойствие
        \item Нужно изучить конфликтную ситуацию, а не принять немедленно решение
        \item Нет возможности или желания решить конфликт в свою пользу
        \item Открытое обсуждение конфликта может ухудшить ситуацию
        \item Подчиненные сами урегулируют конфликт
    \end{enumerate}
    \item \textbf{Компромисс} — перестройка собственного поведения и уступки с учетом точки зрения подчиненных. Очень важно разграничить с сотрудничеством. Этот метод близок, но предполагает более поверхностный уровень. Вы идете на взаимные уступки, но для вас они не имеют принципиального значения, а для противоположной стороны — это важно

    Он используется в следующих случаях:
    \begin{enumerate}
        \item Возможно временное решение, которое впоследствии может быть отменено
        \item Удовлетворение желания партнера имеет не слишком большое значение для вас
        \item Обе конфликтующие стороны имеют одинаково убедительные аргументы
    \end{enumerate}
\end{enumerate}

\subsection{Косвенные, или скрытые, пути разрешения конфликтов}

Помимо прямых, открытых методов разрешения конфликта существуют, может быть, менее заметные, но не менее эффективные скрытые, косвенные методы носят характер опосредованных скрытых воздействий

Как наиболее эффективные специалисты выделяют:

\begin{enumerate}
    \item \textbf{Принцип выхода чувств}. Если человеку дать возможность беспрепятственно выразить свои отрицательные эмоции, то постепенно они сменяются положительными и снимают напряжение в коллективе. Смех, шутка вызывают тот же эффект. Смех пробуждает чувство собственного достоинства, снимает агрессию и заменяет ее интеллектуальным превосходством.
    \item \textbf{Принцип эмоционального возмещения}. Человек, который к вам обращается с жалобами, должен рассматриваться как страдающее лицо. Даже если вам совершенно ясно, что пострадавшим является не он, а именно его недруг. Однако внутреннее ощущение ситуации у вашего собеседника именно его выставляет страдальцем. И чем больше он не прав, тем активнее выгораживает себя перед собственной совестью и выставляет в качестве мученика и жертвы. То, что вы считаетесь с его оскорбленными чувствами, вы тем самым эмоционально подбадриваете своего собеседника.
    \item \textbf{Принцип обнаженной агрессии}. Посредник намеренно предоставляет конфликтным сторонам возможность выразить свою неприязнь друг к другу. Прямая форма обнажения агрессии реализуется следующим образом: в кабинете посредник побуждает партнеров конфликта ссориться в его присутствии. Дав им выговориться, посредник не отпускает их, а продолжает работу. Он предлагает каждому из них, прежде чем ответить оппоненту, повторить его последнюю реплику. Это дает ему право высказаться. Обычно при этом обнаруживается, что ссорящиеся не в состоянии правильно воспроизвести реплики друг друга, поскольку каждый слышит в основном себя, а обидчику приписывает слова, которые в действительности не были произнесены. Фиксируя внимание на этом факте, посредник принуждает их к добросовестному слушанию друг друга. Непривычность такой ситуации уменьшает накал страстей и способствует росту самокритичности.
\end{enumerate}

\pagebreak
\section{Список использованной литературы}

\begin{enumerate}
    \item Кочерина Н. И., Этика деловых отношений и предупреждение конфликтов / Н.И. Кочерина // Труд и социальные отношения. 2010. №10. С. 20-24.
    \item Зиммель Г., Конфликт современной культуры. Избранное. Т.1. Философия культуры / Г. Зиммель. М.: Юристъ, 1996.
    \item  Хасан, Б. И.  Конструктивная психология конфликта : учебное пособие для вузов / Б. И. Хасан. — 2-е изд., стер. — Москва : Издательство Юрайт, 2023. — 204 с. — (Высшее образование). — ISBN 978-5-534-06474-2. — Текст : электронный // Образовательная платформа Юрайт [сайт]. — URL: https://urait.ru/bcode/514511 (дата обращения: 05.04.2023).
    \item Шеламова Г. М., Деловая культура и психология общения / Г. М, Шеламова. – М. : Издательский центр «Академия», 2009. – 192 с. – ISBN 978-5-7695-6466-6.
    \item  Дорошенко В. Ю., Психология и этика делового общения / В. Ю. Дорошенко, Зотова Л. И., Лавриненко В. Н. – Москва : Культура и спорт, ЮНИТИ, 1997. – 279 с. – ISBN 5-85178-046-0.
    \item Кибанов А. Я., Этика деловых отношений / Кибанов А. Я., Захаров Д. К., Коновалова В. Г. — Москва : Инфра-М, 2002. — 368 с. — ISBN: 5-16-001082-3
    \item Кузнецов И. Н., Деловое общение / Кузнецов И. Н. — Москва : Дашков и К, 2013 — 528 с. — ISBN: 978-5-394-01739-1
\end{enumerate}

\end{document}
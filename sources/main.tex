\documentclass{article}
\usepackage[utf8]{inputenc}

\usepackage[T2A]{fontenc}
\usepackage[utf8]{inputenc}
\usepackage[russian]{babel}

\usepackage{multienum}
\usepackage{geometry}
\usepackage{hyperref}

\geometry{
    left=1cm,right=1cm,
    top=2cm,bottom=2cm
}

\title{Конфликты в коллективе}
\author{Сидорук Данил Вадимович}
\date{6 апреля 2023 г.}

\newtheorem{definition}{Определение}

\begin{document}
\raggedright

\maketitle
\tableofcontents
\pagebreak

\section{Реферат по дисциплине «этика делового общения», тема: «конфликты в коллективе»}

\subsection{Введение}

Проблема исследования конфликтов \textbf{является междисциплинарной и весьма актуальной в настоящее время}, как для ученых, так и для практических психологов, социологов, руководителей любого ранга. Причина заключается в том, что \textbf{конфликты возникают постоянно практически во всех сферах человеческой жизни}. Несмотря на многочисленные исследования по вопросам возникновения, функционирования, разрешения конфликтов, попытки классификации конфликтов по различным основаниям, \textbf{общепринятой теории конфликтов, объясняющей их природу, влияние на развитие личности, коллектива, общества, нет}.

\hfill

Учение о конфликтах \textbf{берет свое начало в трудах древнегреческих ученых} — основоположников этики — \textbf{Сократа, Платона, Гераклита}, находит развитие в учении о противоречиях и борьбе противоположностей \textbf{Г. Гегеля}. Впервые попытка рассмотреть конфликт как системное и необходимое явление культуры в целом предпринята \textbf{Г. Зиммелем}, который писал о непрерывной изменчивости культурных явлений и стилей, о противоречии, результатом которого является зримое вытеснение старых форм новыми. Дальнейшие подходы рассматривали понятие социального конфликта как \textbf{борьбы за ценности и притязания на статус, власть и ресурсы, в ходе которой оппоненты нейтрализуют,
наносят ущерб или устраняют своих соперников}.

\subsection{Понятие конфликта, его типы и структура}

\begin{definition}
\textbf{Конфликт} — это открытое противостояние, являющееся следствием взаимоисключающих интересов и позиций
\end{definition}

\textbf{Причины} возникновения конфликта в деловом общении:

\begin{enumerate}
    \item \textbf{Условия} трудового процесса
    \item Психологические \textbf{особенности человеческих взаимоотношений}
    \item \textbf{Корпоративная культура}, деловая этика и традиции данного трудового коллектива
\end{enumerate}

В деловой среде конфликты разделяют на \textbf{конфликты по горизонтали} и на \textbf{конфликты по вертикали}:

\begin{enumerate}
    \item \textbf{Конфликт по горизонтали} возникает между рядовыми сотрудниками одной организации, которые не находятся в подчинении друг другу
    \item \textbf{Конфликт по вертикали} возникает между людьми, находящимися в подчинении друг к другу
\end{enumerate}

\subsubsection{Последствия конфликтов}

\textbf{Негативные} последствия конфликтов:

\begin{multienumerate}
    \mitemxxx{Физическая и духовная усталость}{Ухудшение взаимоотношений}{Снижение функциональности}
\end{multienumerate}

\textbf{Позитивные} последствия конфликтов:

\begin{enumerate}
    \item Конфликт является источником развития
    \item Дает возможность сформулировать и осознать проблему, выявить трудности
    \item Конфликт помогает найти новые пути решения
    \item Учит формам общения, умению налаживать отношения
    \item Оказывает содействие личностному, профессиональному росту и развивает волевую саморегуляцию
    \item Конфликт — это возможность разрядки напряжения, «оздоровления» отношений
\end{enumerate}

\subsection{Способы разрешения конфликтов}

\textbf{Разрешение конфликта} связано с изменением конфликтной ситуации, а \textbf{способы разрешения} — со способами изменения конфликтной ситуации.

Важно помнить, что:

\begin{enumerate}
    \item Неразрешенный конфликт \textbf{порождает новые конфликтные ситуации} с новыми оппонентами
    \item Между оппонентами \textbf{возникает и укрепляется чувство неприязни}, они превращаются в противников
\end{enumerate}

Ученые выделяют два основных типа методов решения конфликтов: \textbf{прямые} (открытые) и \textbf{косвенные} (скрытые).

\subsubsection{Прямые, или открытые, пути разрешения конфликтов}

\begin{enumerate}
    \item \textbf{Директивный — настойчивое утверждение своей точки зрения}. Этот метод не самый эффективный, но зато именно к нему и прибегает большинство людей.

    Он используется лишь в следующих случаях:
    \begin{enumerate}
        \item Вы обладаете несомненной властью и авторитетом, а предлагаемое решение — наилучшее
        \item Вы чувствуете, что нет выбора и вам нечего терять
        \begin{enumerate}
            \item У вас достаточно полномочий, чтобы принять необходимое, но непопулярное решение
            \item Если взаимодействуете с подчиненными, предпочитающими авторитарный стиль
        \end{enumerate}
    \end{enumerate}
    \item \textbf{Сотрудничество - находится наиболее приемлемое для конфликтующих сторон решение, а оппоненты превращаются в партнеров}. Это один из самых сложных путей, но самых эффективных.

    Он используется в следующих случаях:
    \begin{enumerate}
        \item Существуют длительные и прочные отношения между сторонами конфликта
        \item Имеется общая точка зрения на конфликт
    \end{enumerate}
    \item \textbf{Игнорирование конфликта}. Такой путь вовсе не означает вашего поражения. Вы просто \textbf{сглаживаете атмосферу, не пытаясь отстаивать свои интересы, выжидаете более благоприятного момента, чтобы решить свою проблему}

    Он используется в следующих случаях:
    \begin{enumerate}
        \item Источник разногласий тривиален и несущественен по сравнению с более важными задачами
        \item Слишком велика напряженность, и необходимо время, чтобы восстановить спокойствие
        \item Нужно изучить конфликтную ситуацию, а не принять немедленно решение
        \item Нет возможности или желания решить конфликт в свою пользу
        \item Открытое обсуждение конфликта может ухудшить ситуацию
        \item Подчиненные сами урегулируют конфликт
    \end{enumerate}
    \item \textbf{Компромисс} — \textbf{перестройка собственного поведения и уступки с учетом точки зрения подчиненных}. Очень важно разграничить с сотрудничеством. Этот метод близок, но предполагает более поверхностный уровень. Вы идете на взаимные уступки, но для вас они не имеют принципиального значения, а для противоположной стороны — это важно

    Он используется в следующих случаях:
    \begin{enumerate}
        \item Возможно временное решение, которое впоследствии может быть отменено
        \item Удовлетворение желания партнера имеет не слишком большое значение для вас
        \item Обе конфликтующие стороны имеют одинаково убедительные аргументы
    \end{enumerate}
\end{enumerate}

\subsubsection{Косвенные, или скрытые, пути разрешения конфликтов}

Помимо прямых, открытых методов разрешения конфликта существуют, может быть, \textbf{менее заметные, но не менее эффективные скрытые}, косвенные методы носят характер \textbf{опосредованных скрытых воздействий}

Как \textbf{наиболее эффективные} специалисты выделяют:

\begin{enumerate}
    \item \textbf{Принцип выхода чувств}. \textbf{Если человеку дать возможность беспрепятственно выразить свои отрицательные эмоции, то постепенно они сменяются положительными и снимают напряжение в коллективе}. Смех, шутка вызывают тот же эффект. Смех пробуждает чувство собственного достоинства, снимает агрессию и заменяет ее интеллектуальным превосходством.
    \item \textbf{Принцип эмоционального возмещения}. \textbf{Человек, который к вам обращается с жалобами, должен рассматриваться как страдающее лицо}. Даже если вам совершенно ясно, что пострадавшим является не он, а именно его недруг. Однако внутреннее ощущение ситуации у вашего собеседника именно его выставляет страдальцем. И чем больше он не прав, тем активнее выгораживает себя перед собственной совестью и выставляет в качестве мученика и жертвы. То, что вы считаетесь с его оскорбленными чувствами, вы тем самым эмоционально подбадриваете своего собеседника.
    \item \textbf{Принцип обнаженной агрессии}. Посредник намеренно предоставляет конфликтным сторонам возможность выразить свою неприязнь друг к другу. Прямая форма обнажения агрессии реализуется следующим образом: в кабинете посредник побуждает партнеров конфликта ссориться в его присутствии. Дав им выговориться, посредник не отпускает их, а продолжает работу. Он \textbf{предлагает каждому из них, прежде чем ответить оппоненту, повторить его последнюю реплику}. Это дает ему право высказаться. \textbf{Обычно при этом обнаруживается, что ссорящиеся не в состоянии правильно воспроизвести реплики друг друга, поскольку каждый слышит в основном себя, а обидчику приписывает слова, которые в действительности не были произнесены}. Фиксируя внимание на этом факте, \textbf{посредник принуждает их к добросовестному слушанию друг друга}. Непривычность такой ситуации уменьшает накал страстей и способствует росту самокритичности.
\end{enumerate}

\subsection{Список использованной литературы}

\begin{enumerate}
    \item Зиммель Г., Конфликт современной культуры. Избранное. Т.1.Философия культуры / Г. Зиммель. М.: Юристъ, 1996
    \item Хасан Б.И., Конструктивная психология конфликта. СПб.: Питер, 2003
    \item Кочерина Н.И., Этика деловых отношений и предупреждение конфликтов / Н.И. Кочерина // Труд и социальные отношения. 2010. №10. С. 20-24
    \item Шеламова, Г. М, Деловая культура и психология общения / Г. М, Шеламова. – М. : Издательский центр «Академия», 2009. – 192 с. – ISBN 978-5-7695-6466-6.
\end{enumerate}

\end{document}